% ###########################################
% PACKAGES MATHS
% ###########################################
%% Intégrer dans 'mypackage.sty'
% \usepackage{amsthm}
% \usepackage{amsmath}
% \usepackage{amssymb}
% \usepackage{mathrsfs}
% \usepackage{mathtools}
% \usepackage[g]{esvect}
% \usepackage{esint} % For improved integral symbols
% \usepackage{stmaryrd}
% ###########################################
% PACKAGES LAYOUT
% ###########################################
\usepackage{ulem}
\usepackage{lipsum}
\usepackage{easy-todo}
% -------------------------------------------
% RELATED DOC (package ulem) :
% \uline{important} ; \uuline{urgent}
% \uwave{boat} ; \sout{wrong}
% \xout{removed} ; \dashuline{dashing}
% \dotuline{dotty} ; Official doc :
% https://mirrors.chevalier.io/CTAN/macros/latex/contrib/ulem/ulem.pdf 
% Snippet pour visualiser la doc : ulemdoc
% -------------------------------------------
\usepackage{stackengine}
\usepackage{scalerel}
\usepackage{lscape}
\usepackage{calc}
\usepackage[full]{textcomp}
\usepackage{lastpage}
% Pour les tableaux sur une page en landscape
\usepackage{pdflscape}
\usepackage{afterpage}
\usepackage{pdfpages} % Pour inclure des documents en pdf
% -------------------------------------------
% RELATED DOC (package setspace) :
% \singlespacing ; \onehalfspacing ; \doublespacing
% -------------------------------------------
\usepackage{numprint}
\usepackage{hyperref}
\usepackage{float}
\usepackage{pdfpages}
\usepackage{numprint}
% Pour les environnements flottants
\usepackage{newfloat}
\usepackage{caption}
% ###########################################
% PACKAGES LAYOUT TABLEAUX
% ###########################################
\usepackage{etoolbox}
% Réglage de l'espace avant et après le tableau :
\BeforeBeginEnvironment{tabular}{\medskip}
\AfterEndEnvironment{tabular}{\medskip}
\usepackage{array}
\usepackage{booktabs}
\usepackage{cellspace}
\usepackage{makecell}
\newcommand\xrowht[2][0]{\addstackgap[.5\dimexpr#2\relax]{\vphantom{#1}}}
\usepackage{arydshln}
\newcommand\VRule[1][\arrayrulewidth]{\vrule width #1}
% ###########################################
% PACKAGES POLICE + SYMBOLES
% ###########################################
\usepackage{fontawesome}
\usepackage{stmaryrd}
% ###########################################
% PACKAGE DESSIN (Tikz)
% ###########################################
\usepackage{xcolor}
\usepackage{tikz}
\usepackage{pgfplots}
\usepackage{colortbl}
\usepackage[most]{tcolorbox} % For better colored and fancy boxes
\usepackage[tikz]{bclogo}
\usepackage{circledsteps} % For circled letters
\usepackage{graphicx}
\usepackage{wrapfig}
\usepackage{fancybox}
\usepackage{mdframed} 
% Libraries Tikz :
\usetikzlibrary{decorations.pathmorphing}
\usetikzlibrary{decorations.pathreplacing}
\usetikzlibrary{decorations.shapes}
\usetikzlibrary{decorations.text}
\usetikzlibrary{angles}
\usetikzlibrary{decorations.markings}
\usetikzlibrary{decorations.fractals}
\usetikzlibrary{decorations.footprints}
\usetikzlibrary{patterns}
\usetikzlibrary{plotmarks}
% ----------------------------
% Packages pour présenter du code
% ----------------------------
\usepackage{listings}

\lstdefinestyle{pprint}{
language=Python,
breaklines=true,
basicstyle=\footnotesize\ttfamily,
keywordstyle=\bfseries\color{green!40!black},
commentstyle=\itshape\color{gray!40},
stringstyle=\color{violet},
numberstyle=\tiny,
keepspaces=true,
numbers=left,
%frame=single,
%framesep=2pt,
aboveskip=1ex,
showtabs=true,
captionpos=b
} % Python

% Pour écrire des algorithmes en symbolique
\usepackage{algorithm}
\usepackage{algorithmic}