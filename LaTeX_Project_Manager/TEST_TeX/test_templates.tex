%ID:thm-1-1-1
%TITLE:Continuité monotone
%BEGIN
Soit $(E,\frakT,\mu)$ un espace mesuré. On note $(E_n)_{n\in\doubleN}$ avec $n\in\doubleN$, une famille d'espaces mesurables. Alors on a :
\begin{enumerate}
\item \textbf{Continuité croissante:} Si $E_n\subset E_{n+1}$, alors
\begin{equation*}
\lim\limits_{n \rightarrow +\infty} \mu(E_n) = \mu\left(\bigcup_{n\in\doubleN} E_n\right)
\end{equation*}
\item \textbf{Continuité décroissante:} Si $E_{n+1}\subset E_n$, alors :
\begin{equation*}
\lim\limits_{n \rightarrow +\infty} \mu(E_n) = \mu\left(\bigcap_{n\in\doubleN} E_n\right)
\end{equation*}
%END
%ID:thm-1-2-2
%TITLE:Propriétés
%BEGIN
Soit $(E,\frakT,\mu)$, un espace mesuré. On dispose des propriétés suivantes
\begin{enumerate}
\item La mesure $\mu$ est \textbf{croissante} sur $\frakT$ (\textsl{au sens de l'inclusion}), c'est-à-dire $\forall (A,B)\subset \frakT^2,~A\subset B\Rightarrow \mu(A)\leq \mu(B)$
\item si $\mu(A)<+\infty$, alors $\mu(B\setminus A)=\mu(B)-\mu(A)$
\item On dispose également de la propriété suivante : $\forall (A,B)\subset \frakT^2,~\mu(A\cup B) + \mu(A\cap B) = \mu(A) + \mu(B)$
\item Soit $\scriptI$, un ensemble au plus dénombrable et $(A_i)_{i\in\scriptI}\in\frakT$, alors :
\begin{equation*}
\mu\left(\bigcup_{i\in\scriptI}A_i\right) \leq\sum_{i\in\scriptI}\mu(A_i)\tag{Sous-additivité}
\end{equation*}
\end{enumerate}
%END
%ID:thm-2-1-3
%TITLE:Mesure de \textsc{Lebesgue}
%BEGIN
Soit $d\in\doubleN^*$. Il existe une \textbf{unique} mesure $\lambda_d$ sur $(\doubleR^d,\scriptB(\doubleR^d))$ telle que $\lambda_d([0,1]^d)=1$ et telle que $\lambda_d$ soit \textbf{invariante par translation}, c'est-à-dire $\forall x\in\doubleR$, $\forall B\in\scriptB(\doubleR^d),$ $\lambda_d(B+x) = \lambda_d(B)$. On l'appelle $\lambda_d$ la mesure de \textsc{Lebesgue} $d$-dimensionnelle.
%END
%ID:thm-2-2-4
%TITLE:Applications mesurables
%BEGIN
Soit $(E,\frakT,\mu)$ et $(F,\frakU,\nu)$, deux espaces mesurés, et $f\colon E\to F$ une application. On dira que $f$ est $(\frakT,\frakU)$\textbf{-mesurable}, ou simplement \textbf{mesurables}, si et seulement si 
\begin{equation*}
\forall A\in\frakU,~f^{-1}(A)\in\frakT
\end{equation*}
$f^{-1}(A)$ est la \textbf{tribu image réciproque}\footnote{De même, on appelle \textbf{tribu image} de $\frakT$ par une application $f$ la tribu sur $\frakU$ notée $f(\frakT) = \lbrace B\subset \frakU,~f^{-1}(B)\in\frakT\rbrace$} par $f$ de $\frakU$.