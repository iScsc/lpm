% ===========================================
% Définition de macros complexes
% ===========================================
\begingroup
    \def\tmpa#1{%
        \if\relax#1 \expandafter\noexpand \else
            \expandafter\gdef\csname double#1\endcsname{\mathbb #1} %
            \expandafter\gdef\csname script#1\endcsname{\mathcal #1} %
			\expandafter\gdef\csname frak#1\endcsname{\mathfrak #1} %
            \expandafter\gdef\csname cal#1\endcsname{\mathscr #1} %
        \fi \tmpa
    }
    \tmpa ABCDEFGHIJKLMNOPQRSTUVWXYZabcdefghijklmnopqrstuvwxyz\relax
\endgroup
%====================================================
%####################################################
%====================================================
\def\restriction#1#2{\mathchoice
              {\setbox1\hbox{${\displaystyle #1}_{\scriptstyle #2}$}
              \restrictionaux{#1}{#2}}
              {\setbox1\hbox{${\textstyle #1}_{\scriptstyle #2}$}
              \restrictionaux{#1}{#2}}
              {\setbox1\hbox{${\scriptstyle #1}_{\scriptscriptstyle #2}$}
              \restrictionaux{#1}{#2}}
              {\setbox1\hbox{${\scriptscriptstyle #1}_{\scriptscriptstyle #2}$}
              \restrictionaux{#1}{#2}}}
\def\restrictionaux#1#2{{#1\,\smash{\vrule height .8\ht1 depth .85\dp1}}_{\,#2}}
\newcommand{\de}[2]{\frac{\mathrm{d}#1}{\mathrm{d}#2}} % dérivée
\newcommand{\vect}[2]{\vv{#1_{#2}}} % vecteur de 
\newcommand{\dep}[1]{\frac{\partial}{\partial#1}} % dérivée partielle nabla
\newcommand{\rot}[1]{\vv{\mathrm{rot}}\left(#1\right)}
\newcommand{\grade}[1]{\vv{\mathrm{grad}}\left(#1\right)}
\newcommand{\dive}[1]{\mathrm{div}\left(#1\right)}
% à remplacer par un snippet
\newcommand{\func}[5]{
    \begin{align*}
    #1\colon #2 & \longrightarrow #3\\
    #4&\longmapsto #5
    \end{align*}  
}
% ===========================================
% Définition d'opérateurs mathématiques
% ===========================================
\DeclareMathOperator{\arccot}{arccot}
\DeclareMathOperator{\arcsec}{arcsec}
\DeclareMathOperator{\arccsc}{arccsc}
\DeclareMathOperator{\sech}{sech}
\DeclareMathOperator{\csch}{csch}
\DeclareMathOperator{\argsh}{argsh}
\DeclareMathOperator{\argth}{argth}
\DeclareMathOperator{\argch}{argch}
\DeclareMathOperator{\argcoth}{argcoth}
\DeclareMathOperator{\arsech}{arsech}
\DeclareMathOperator{\arcsch}{arcsch}
\DeclareMathOperator{\rg}{rg}
\DeclareMathOperator{\card}{card}
\DeclareMathOperator{\Sp}{Sp}
% ===========================================
% Définition des environnements mathématiques
% ===========================================
\theoremstyle{plain}
\newtheorem{theorem}{Théorème}

\theoremstyle{definition}
\newtheorem{proposition}{Proposition}
\newtheorem{definition}{Définition}
\newtheorem{corollary}{Corollaire}

\theoremstyle{remark}
\newtheorem*{remark}{Remarque}
\newtheorem{lemma}{Lemme}