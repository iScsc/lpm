%% DEPRECATED: complétement intégré dans 'mypackage.sty'.
% ===========================================
% Définition de macros complexes
% ===========================================
\begingroup
    \def\tmpa#1{%
        \if\relax#1 \expandafter\noexpand \else
            \expandafter\gdef\csname double#1\endcsname{\mathbb #1} %
            \expandafter\gdef\csname script#1\endcsname{\mathcal #1} %
			\expandafter\gdef\csname frak#1\endcsname{\mathfrak #1} %
            \expandafter\gdef\csname cal#1\endcsname{\mathscr #1} %
        \fi \tmpa
    }
    \tmpa ABCDEFGHIJKLMNOPQRSTUVWXYZabcdefghijklmnopqrstuvwxyz\relax
\endgroup
%====================================================
%####################################################
%====================================================
\def\restriction#1#2{\mathchoice
              {\setbox1\hbox{${\displaystyle #1}_{\scriptstyle #2}$}
              \restrictionaux{#1}{#2}}
              {\setbox1\hbox{${\textstyle #1}_{\scriptstyle #2}$}
              \restrictionaux{#1}{#2}}
              {\setbox1\hbox{${\scriptstyle #1}_{\scriptscriptstyle #2}$}
              \restrictionaux{#1}{#2}}
              {\setbox1\hbox{${\scriptscriptstyle #1}_{\scriptscriptstyle #2}$}
              \restrictionaux{#1}{#2}}}
\def\restrictionaux#1#2{{#1\,\smash{\vrule height .8\ht1 depth .85\dp1}}_{\,#2}}
% ===========================================
% Définition d'opérateurs mathématiques
% ===========================================
% Fonctions trigonométriques (circulaires et hyperboliques)
\DeclareMathOperator{\arccot}{arccot}
\DeclareMathOperator{\arcsec}{arcsec}
\DeclareMathOperator{\arccsc}{arccsc}
\DeclareMathOperator{\sech}{sech}
\DeclareMathOperator{\csch}{csch}
\DeclareMathOperator{\argsh}{argsh}
\DeclareMathOperator{\argth}{argth}
\DeclareMathOperator{\argch}{argch}
\DeclareMathOperator{\argcoth}{argcoth}
\DeclareMathOperator{\arsech}{arsech}
\DeclareMathOperator{\arcsch}{arcsch}
% Opérateurs divers
\DeclareMathOperator{\rg}{rg}
\DeclareMathOperator{\card}{card}
\DeclareMathOperator{\Sp}{Sp}
\DeclareMathOperator{\tr}{tr}
% Définitions des opérateurs vectoriels :
\DeclareMathOperator{\dev}{dev}
\newcommand{\ull}[1]{\underline{\underline{#1}}}
\DeclareMathOperator{\Div}{div}
\newcommand{\divv}{\underline{\Div}}
\newcommand{\grad}{\vv{\mathrm{grad}}}
\newcommand{\gradt}{\ull{\mathrm{grad}}}
\newcommand{\rot}{\vv{\mathrm{rot}}}
% ===========================================
% Définition des environnements mathématiques
% ===========================================
\theoremstyle{plain}
\newtheorem{theorem}{Théorème}

\theoremstyle{definition}
\newtheorem{proposition}{Proposition}
\newtheorem{definition}{Définition}
\newtheorem{corollary}{Corollaire}

\theoremstyle{remark}
\newtheorem*{remark}{Remarque}
\newtheorem{lemma}{Lemme}
% ###########################################
% MACROS TEXTE
% ###########################################
\newcounter{newquestion}
\setcounter{newquestion}{1}
\newcounter{newsubquestion}
\setcounter{newsubquestion}{1}
\newcommand{\myquestion}[2]{
    \noindent\textbf{\thenewquestion.~-}~\textsl{#1}\newline{\color{red}#2}\newline
    \stepcounter{newquestion}}
\newcommand{\mysubquestion}[2]{
    {\color{black}\textbf{\thenewquestion(\romannumeral\thenewsubquestion).~-}~#1}\newline{\color{red}#2}
    \newline\stepcounter{newsubquestion}}